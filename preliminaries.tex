\documentclass{article}

\usepackage{amsmath}
\usepackage{unicode-math}

\DeclareMathOperator{\Bij}{Bij}
\renewcommand{\k}{\mathbb{k}}

\title{Towards manifolds}

\author{Patrick Massot}

\begin{document}
\maketitle

\section{Algebraic preliminaries}
\label{sec:algebraic_preliminaries}

A left action of a group $G$ on a set $X$ is a group morphism $ρ$ from $G$
to the group of permutation $\Bij(X)$. The orbit $Gx$ of a point $x ∈ X$
under $G$ is $\{ ρ(g)(x); g ∈ G\}$. An action is transitive if 
$Gx = X$ for some (or equivalently all) $x$ in $X$. An action $ρ$ is
called free when it is injective. A principal $G$-space is a set
equipped with a transitive free action of $G$.

An affine space over a field $\k$ is $E$-principal space $X$ for some
$k$-vector space $E$. We also say that $X$ is directed by $E$.
Notations: the action of $v ∈ E$ on a point 
$x ∈ X$ is denoted by $x + v$. Given $x$ and $y$, the unique $v$ in $E$
such that $y = x + v$ is denoted by $y - x$.

\section{Topological preliminaries}
\label{sec:topological_preliminaries}

A topological space is Hausdorff if any two points have disjoint
neighborhoods. It is second-countable if its topology has a countable
basis.

A topological vector space $E$ over a topological field $\k$ is a
$\k$-vector space equipped with a topology such that addition is
continuous from $E × E$ to $E$ and multiplication is continuous from 
$\k × E$ to $E$.

For every affine space directed by a topological vector space, there is
a unique topology such that each $v ↦ x + v$ is a homeomorphism.

Let $\k$ be a field. An absolute value on $\k$ is a map $|·|$ from 
$\k$ to $ℝ$ such that $|x| = 0 ⇔ x = 0$, $|xy|= |x|·|y|$, 
$|x + y| ≤ |x| + |y|$.
A norm on a $\k$-vector space $E$ with absolute value is a map $‖·‖$
from $E$ to $ℝ$ such $‖x‖ = 0 ⇔ x = 0$, $‖x + y‖ ≤ ‖x‖ + ‖y‖$, 
$‖λ x‖ =|λ| · ‖x‖$.

Any two norms $‖·‖₁$ and $‖·‖₂$ on a finite dimensional real
vector space are equivalent: there exists a positive real number $C$
such that: $‖·‖₁/C ≤ ‖·‖₂ ≤ C‖·‖₁$ (this is not quite trivial, requires
compactness…).

To each norm $‖·‖$ one can associate a distance $d(x, y) = ‖x - y‖$, and
then associated topology. In the case of finite dimensional vector
spaces, the resulting topology is independent of the chosen norm (by
equivalence of norms), and all linear maps are continuous.

Norm on continuous linear maps spaces $L(E, F)$: 
$‖φ‖ = \max \{‖φx‖ ; ‖x‖ = 1\}$. This makes $L(E, F)$ into a normed
algebra (normed vector space and $‖φ ∘ ψ‖ ≤ ‖φ‖·‖ψ‖$). 

\section{Differential calculus}
\label{sec:differential_calculus}

Let $X$ and $Y$ be two affine space directed by normed vector spaces
$E$ and $F$. Let $U$ be an open subset of $X$. A map $f : U → Y$ is
differentiable at $x ∈ U$ if there exists a continuous linear map 
$L : E → F$ such that, for sufficiently small $v$,
$f(x + v) = f(x) + Lv + ‖v‖ε(v)$ for some function $ε : E → F$ which
goes to zero when $v$ goes to $0$ (let's pretend it would be reasonable
not to define $o(v)$ notation). In this case, $L$ is unique, and denote
by $Df(x)$. The map $f$ is differentiable on $X$ if it is differentiable
at every $x$ in $X$. By induction, define $(k+1)$-times differentiable
maps by $Df$ is $k$-times differentiable.
It is $C¹$ if $x ↦ Df(x)$ is continuous. By induction, a map is
$C^{k+1}$ if $Df$ is $C^k$. It is smooth, or $C^∞$ if it is $C^k$ for
all $k$. 

The chain rule. Let $X$, $Y$ and $Z$ be affine spaces. Let $U$ and $V$
be open subsets in $X$ and $Y$ respectively. If $f : U → V$ is
differentiable at $x$ and $g : V → Z$ is differentiable at $f(x)$ then
$g ∘ f$ is differentiable at $x$, and $D(g ∘ f)(x) = Dg(f(x)) ∘ Df(x)$.

A composition of two $C^k$ maps is $C^k$ (but let's not state the
formula for the higher order differential).

Almost any serious proof in differential topology will rely on the
inverse function theorem, but not the early definitions so let's omit it
at first.
\end{document}
